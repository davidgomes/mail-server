\documentclass[12pt]{article}

\usepackage[utf8]{inputenc}
\usepackage{listings}
\usepackage{graphicx}
\usepackage{float}
\usepackage{geometry}

\usepackage{parskip}
\setlength{\parskip}{1.0\baselineskip plus2pt minus2pt}

\addtolength{\topmargin}{-50pt}
\addtolength{\textheight}{130pt}
\addtolength{\textwidth}{95pt}
\addtolength{\oddsidemargin}{-45pt}

\title{Cliente e Servidor de Correio Eletrónico}
\author{David Gomes (2013136061), Nuno Gonçalves (2013140672)}
\date{Dezembro 2014}

\begin{document}
\maketitle

Neste relatório descrevemos o nosso segundo projeto de Introdução às Redes
e Comunicações, que envolvia criar um cliente e um servidor de email com
recurso a \textit{sockets}. Implementamos tanto o cliente como o servidor
em Python, de forma a simplificar o envio de informação entre cliente e servidor.

\section{Comunicação Cliente-Servidor}
Para comunicarmos entre o cliente e o servidor definimos três comandos.

\subsection{GET}
O comando \texttt{GET} recebe apenas um argumento que é um dicionário da forma:

\vspace{2mm}
\begin{lstlisting}[language=Python]
  {
    'user': String,
    'password': String
  }
\end{lstlisting}

Este comando, se sucedido, vai devolver um dicionário da forma:

\vspace{2mm}
\begin{lstlisting}[language=Python]
  {
    'user': String,
    'password': String,

    'emails': {
      'sent': [],
      'received': []
    }
  }
\end{lstlisting}

Basicamente, o comando \texttt{GET} serve para efetuar o \textit{login} no
servidor.

\subsection{SEND}
O comando \texttt{SEND} recebe um argumento que é um dicionário da forma:

\vspace{2mm}
\begin{lstlisting}[language=Python]
  {
    'receivers': [],
    'subject': String,
    'content': String
  }
\end{lstlisting}

Este comando devolve \textit{True} ou \textit{False} caso o envio da
mensagem tenha sucedido ou não.

\subsection{DELETE}
O comando \texttt{DELETE} recebe um argumento que é um dicionário da forma:

\vspace{2mm}
\begin{lstlisting}[language=Python]
  {
    'receivers': [],
    'subject': String,
    'content': String
  }
\end{lstlisting}

Este comando devolve \textit{True} ou \textit{False} caso a remoção da
mensagem tenha sucedido ou não.

\subsection{Detalhes}
Tanto os comandos \texttt{SEND} como o comando \texttt{DELETE} atuam sobre o utilizador
ligado à \textit{socket} atual.

O nosso programa permite listar e ver qualquer uma das mensagens recebidas ou
enviadas, assim como apagar qualquer uma destas.

Finalmente, escrevemos num ficheiro de texto em JSON a informação e as mensagens
de todos os utilizadores, atualizada quando terminamos o processo do servidor.

\section{Comunicação Servidor-Servidor}
A nossa interpretação na questão do duplo servidor foi criar um \texttt{server2}
que atua como servidor de \textit{backup} ao servidor principal. Quando um cliente
se encontra ligado ao servidor principal e tenta enviar uma mensagem a um cliente
que não se encontra registado neste servidor, o servidor tentará enviar a mensagem
pelo segundo servidor. No entanto, esta funcionalidade só se verifica caso o cliente
que envia a mensagem também esteja registado no servidor de \textit{backup}.

\end{document}
